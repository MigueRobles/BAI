\documentclass[12pt,twoside]{article}
%%%%%%%%%%%%%%%%%%%%%%%%%%%%%%%%%%%%%%%%%%%%%%%%%%%%%%%%%%%%%
% Meta informations:
\newcommand{\trauthor}{Miguel Angel Robles Urquiza | Rafael Ruz Gomez}
\newcommand{\trtype}{Seminar Paper} %{Seminararbeit} %{Proseminararbeit}
\newcommand{\trcourse}{Bio - Inspired Artificial Intelligence}
\newcommand{\trtitle}{Machine Learning in Chatbots}
\newcommand{\trmatrikelnummer}{7023522 | 7019703}
\newcommand{\tremail}{migueurquiza@gmail.com | rafaroco96@gmail.com}
\newcommand{\trarbeitsbereich}{Knowledge Technology, WTM}
\newcommand{\trdate}{22-12-2017}

%%%%%%%%%%%%%%%%%%%%%%%%%%%%%%%%%%%%%%%%%%%%%%%%%%%%%%%%%%%%%
% Languages:

% Falls die Ausarbeitung in Deutsch erfolgt:
% \usepackage[german]{babel}
% \usepackage[T1]{fontenc}
% \usepackage[latin1]{inputenc}
% \usepackage[latin9]{inputenc}	 				
% \selectlanguage{german}

% If the thesis is written in English:
\usepackage[english]{babel} 						
\selectlanguage{english}

%%%%%%%%%%%%%%%%%%%%%%%%%%%%%%%%%%%%%%%%%%%%%%%%%%%%%%%%%%%%%
% Bind packages:
\usepackage{acronym}                    % Acronyms
\usepackage{algorithmic}								% Algorithms and Pseudocode
\usepackage{algorithm}									% Algorithms and Pseudocode
\usepackage{amsfonts}                   % AMS Math Packet (Fonts)
\usepackage{amsmath}                    % AMS Math Packet
\usepackage{amssymb}                    % Additional mathematical symbols
\usepackage{amsthm}
\usepackage{booktabs}                   % Nicer tables
%\usepackage[font=small,labelfont=bf]{caption} % Numbered captions for figures
\usepackage{color}                      % Enables defining of colors via \definecolor
\definecolor{uhhRed}{RGB}{254,0,0}		  % Official Uni Hamburg Red
\definecolor{uhhGrey}{RGB}{122,122,120} % Official Uni Hamburg Grey
\usepackage{fancybox}                   % Gleichungen einrahmen
\usepackage{fancyhdr}										% Packet for nicer headers
%\usepackage{fancyheadings}             % Nicer numbering of headlines

%\usepackage[outer=3.35cm]{geometry} 	  % Type area (size, margins...) !!!Release version
%\usepackage[outer=2.5cm]{geometry} 		% Type area (size, margins...) !!!Print version
%\usepackage{geometry} 									% Type area (size, margins...) !!!Proofread version
\usepackage[outer=3.15cm]{geometry} 	  % Type area (size, margins...) !!!Draft version
\geometry{a4paper,body={5.8in,9in}}

\usepackage{graphicx}                   % Inclusion of graphics
%\usepackage{latexsym}                  % Special symbols
\usepackage{longtable}									% Allow tables over several parges
\usepackage{listings}                   % Nicer source code listings
\usepackage{multicol}										% Content of a table over several columns
\usepackage{multirow}										% Content of a table over several rows
\usepackage{rotating}										% Alows to rotate text and objects
\usepackage[hang]{subfigure}            % Allows to use multiple (partial) figures in a fig
%\usepackage[font=footnotesize,labelfont=rm]{subfig}	% Pictures in a floating environment
\usepackage{tabularx}										% Tables with fixed width but variable rows
\usepackage{url,xspace,boxedminipage}   % Accurate display of URLs
\usepackage{verbatim}                   % Comments

%%%%%%%%%%%%%%%%%%%%%%%%%%%%%%%%%%%%%%%%%%%%%%%%%%%%%%%%%%%%%
% Configurationen:

\hyphenation{whe-ther} 									% Manually use: "\-" in a word: Staats\-ver\-trag

%\lstloadlanguages{C}                   % Set the default language for listings
\DeclareGraphicsExtensions{.pdf,.svg,.jpg,.png,.eps} % first try pdf, then eps, png and jpg
\graphicspath{{./src/}} 								% Path to a folder where all pictures are located
\pagestyle{fancy} 											% Use nicer header and footer

% Redefine the environments for floating objects:
\setcounter{topnumber}{3}
\setcounter{bottomnumber}{2}
\setcounter{totalnumber}{4}
\renewcommand{\topfraction}{0.9} 			  %Standard: 0.7
\renewcommand{\bottomfraction}{0.5}		  %Standard: 0.3
\renewcommand{\textfraction}{0.1}		  	%Standard: 0.2
\renewcommand{\floatpagefraction}{0.8} 	%Standard: 0.5

% Tables with a nicer padding:
\renewcommand{\arraystretch}{1.2}

%%%%%%%%%%%%%%%%%%%%%%%%%%%%
% Additional 'theorem' and 'definition' blocks:
\theoremstyle{plain}
\newtheorem{theorem}{Theorem}[section]
%\newtheorem{theorem}{Satz}[section]		% Wenn in Deutsch geschrieben wird.
\newtheorem{axiom}{Axiom}[section] 	
%\newtheorem{axiom}{Fakt}[chapter]			% Wenn in Deutsch geschrieben wird.
%Usage:%\begin{axiom}[optional description]%Main part%\end{fakt}

\theoremstyle{definition}
\newtheorem{definition}{Definition}[section]

%Additional types of axioms:
\newtheorem{lemma}[axiom]{Lemma}
\newtheorem{observation}[axiom]{Observation}

%Additional types of definitions:
\theoremstyle{remark}
%\newtheorem{remark}[definition]{Bemerkung} % Wenn in Deutsch geschrieben wird.
\newtheorem{remark}[definition]{Remark} 

%%%%%%%%%%%%%%%%%%%%%%%%%%%%
% Provides TODOs within the margin:
\newcommand{\TODO}[1]{\marginpar{\emph{\small{{\bf TODO: } #1}}}}

%%%%%%%%%%%%%%%%%%%%%%%%%%%%
% Abbreviations and mathematical symbols
\newcommand{\modd}{\text{ mod }}
\newcommand{\RS}{\mathbb{R}}
\newcommand{\NS}{\mathbb{N}}
\newcommand{\ZS}{\mathbb{Z}}
\newcommand{\dnormal}{\mathit{N}}
\newcommand{\duniform}{\mathit{U}}

\newcommand{\erdos}{Erd\H{o}s}
\newcommand{\renyi}{-R\'{e}nyi}
%%%%%%%%%%%%%%%%%%%%%%%%%%%%%%%%%%%%%%%%%%%%%%%%%%%%%%%%%%%%%
% Document:
\begin{document}
\renewcommand{\headheight}{14.5pt}

\fancyhead{}
\fancyhead[LE]{ \slshape \trauthor}
\fancyhead[LO]{}
\fancyhead[RE]{}
\fancyhead[RO]{ \slshape \trtitle}

%%%%%%%%%%%%%%%%%%%%%%%%%%%%
% Cover Header:
\begin{titlepage}
	\begin{flushleft}
		Universit\"at Hamburg\\
		Department Informatik\\
		\trarbeitsbereich\\
	\end{flushleft}
	\vspace{3.5cm}
	\begin{center}
		\huge \trtitle\\
	\end{center}
	\vspace{3.5cm}
	\begin{center}
		\normalsize\trtype\\
		[0.2cm]
		\Large\trcourse\\
		[1.5cm]
		\Large \trauthor\\
		[0.2cm]
		\normalsize Matr.Nr. \trmatrikelnummer\\
		[0.2cm]
		\normalsize\tremail\\
		[1.5cm]
		\Large \trdate
	\end{center}
	\vfill
\end{titlepage}

	%backsite of cover sheet is empty!
\thispagestyle{empty}
\hspace{1cm}
\newpage

%%%%%%%%%%%%%%%%%%%%%%%%%%%%
% Abstract:

% Abstract gives a brief summary of the main points of a paper:
\section*{Abstract}

In this paper we are going to do a research on the automatic learning of the chatbots, focus in sentimental chatbots. First, we will study what is a chatbot, studying its origins and seeing its applications. Once this is done, we will see what are the different types of chatbots that exist today, seeing why we need them, how they work and their advantages and disadvantages.

Once we have introduced the different types of chatbots that exist, we will establish similarities and differences in terms of structure, functionality and use between that chatbots. We will also make an research of the chatbots that because of their low usage and complex and not maintenance have become obsolete, an research of the chatbots that are used to a greater extent nowadays and an research of the chatbots that do not exist yet but we will need in our lives in a future.


\begin{comment}

\end{comment}

% Lists:
\setcounter{tocdepth}{2} 					% depth of the table of contents (for Seminars 2 is recommented)
\tableofcontents

\pagenumbering{arabic}
\clearpage

%%%%%%%%%%%%%%%%%%%%%%%%%%%%
% Content:

% the actual content, usually separated over a number of sections
% each section is assigned a label, in order to be able to put a
% crossreference to it

\section{Introduction}
\label{sec:introduction}
Let's start by defining what chatbots are.\textit{"Chatbots are computer programs that interact with users using natural languages."} \cite{shawar2007chatbots}
A chatbot is a conversational bot whose objetive is to interract with the user, asking question or giving answers or even initiating new topics of conversation. \cite{huang2007extracting}



\section{A Brief Review of ChatBots}
\label{sec:basics}

The need to create virtual machines that speak as if they were human beings arose in the decade of the 60s. ELIZA was born, by the hand of Joseph Weizenbaum, professor emeritus of Informatics at the Massachusetts Institute of Technology. At that time, this chatbot was created to emulate a psychotherapist in clinical treatment.The objective of this teacher when he created the program was simple and made use of keyword matching. A keyword is searched in the input, and if this keyword is in the database, a previously programmed response is answered, and if this does not happen, the program returns a random phrase to continue obtaining information. For example, if the input includes the keyword "mother", ELIZA can respond "Tell me more about your family" \cite{shawar2007chatbots}.

\section{Types of chatbots}
\label{sec:types}

	\subsection{Sentimental-chatbots}
	\label{sec:sentimental}
	
	Many times we need someone to tell us something to make us feel better, someone to be with in bad times or someone to share happiness with. But if we do not have anyone to talk to and we do not want to be "alone" or "sad", what can we do? In this document, we propose a solution. We develop a sentimental chatbot whose objective is to detect the polarity of the feeling of a message. By using this chatbot, you can engage in a conversation that detects whether you are happy or sad, according to your messages and responds appropriately to your messages.

	\subsection{Information-chatbots}
	\label{sec:information}

	\subsection{Game-chatbots}
	\label{sec:game}

	\subsection{Assistant-chatbots}
	\label{sec:assistant}
	

\section{Comparison of chatbots}
\label{sec:comparison}


\section{Chatbots in time}
\label{sec:time}

\subsection{Obsolets}
\label{sec:obsolet}

\subsection{Nowdays}
\label{sec:nowdays}

\subsection{Future}
\label{sec:future}

%%%%%%%%%%%%%%%%%%%%%%%%%%%%%%%%%%%%%%
% hier werden - zum Ende des Textes - die bibliographischen Referenzen
% eingebunden
%
% Insbesondere stehen die eigentlichen Informationen in der Datei
% ``bib.bib''
%
\nocite{*}
\newpage
\bibliographystyle{alpha}
\addcontentsline{toc}{section}{Bibliography}% Add to the TOC
\bibliography{bib}

\end{document}


